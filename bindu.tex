\documentclass[10pt]{memoir}
\setstocksize{220mm}{155mm} 	        
\settrimmedsize{220mm}{155mm}{*}	
\settypeblocksize{170mm}{116mm}{*}	
\setlrmargins{18mm}{*}{*}
\setulmargins{*}{*}{1.2}
%\setlength{\headheight}{5pt}
\checkandfixthelayout[lines]
\linespread{1.16}

%%% Hyphenation settings
\usepackage{hyphenat}
\hyphenation{he-lio-trope opos-sum}
\tracingparagraphs=1
%Hyphenation in Devanāgarī of the edition still missing? Probably this needs to be modified in babel-iast package? 

%%% babel
\usepackage[english]{babel}
\usepackage{babel-iast/babel-iast}
\babelfont[iast]{rm}[Renderer=Harfbuzz, Scale=1.3]{AdishilaSan}%AdishilaSan}
\babelfont[english]{rm}{Adobe Text Pro}

%%% footnote without a marker
%\newcommand\blfootnote[1]{%
%\begingroup
%\renewcommand\thefootnote{}\footnote{\noindent #1}%
%\addtocounter{footnote}{-1}%
%\endgroup
%}

\makeatletter
\def\blfootnote{\gdef@thefnmark{}@footnotetext}
\makeatother

%\renewcommand{\blfootnote}{\arabic{footnote})}

%\makeatletter
%5\def@makefntext#1{%
%\noindent
%\makebox[0.75em][l]{@thefnmark.}%
%\hspace{0.25em}#1}
%\makeatother

%\usepackage{etoolbox}
%\newcounter{blfootnote}
%\newenvironment{blfootnote}[1][]
%{\refstepcounter{blfootnote}%
%\begingroup%
%\def\thefootnote{\arabic{blfootnote}}%
%\ifstrempty{#1}{}{\noindent\hangindent=0.75em\hangafter=1 #1}\footnotetext}
%{\endgroup}
  
%%% more functionality
\PassOptionsToPackage{hyphens}{url}
\usepackage{hyperref}
\usepackage{cleveref}
\usepackage{url}
\usepackage{cleveref}
\usepackage{microtype}
\usepackage{lineno}
%\usepackage{bigfoot}
%%% more functions
\usepackage[dvipsnames]{xcolor}
%\usepackage[flushmargin]{footmisc}

%%%für den Counter von Kapiteln und Sätzen! 
\newcommand{\uproman}[1]{\uppercase\expandafter{\romannumeral#1}}
\newcommand{\lowroman}[1]{\romannumeral#1\relax}

\makeindex
\newfontfamily\sanskritfont[Script=Devanagari,Mapping=RomDev,Scale=1.1]{Sanskrit2003}
\usepackage{pifont,fourier-orns,lettrine,psvectorian,paralist,enumitem,pdfpages,wrapfig,tabulary,lettrine,longtable}
\setlist[enumerate]{itemsep=0mm}
\usepackage[autostyle]{csquotes}
\usepackage[defaultlines=2,all]{nowidow}
\usepackage{ellipsis,adforn,booktabs,longtable,url,tikz}
\lineskiplimit=-3pt          

\makechapterstyle{IeT}{%
  \chapterstyle{default}
  \renewcommand*{\printchapternonum}{\centering}
  \renewcommand*{\clearforchapter}{\cleartorecto} 
  \aliaspagestyle{chapter}{empty}}
\chapterstyle{IeT}
\setsecnumdepth{none}  \openright  \nouppercaseheads
\settocdepth{subsubsection}

%%%% test better pagebreaks
\def\fussy{%
  \emergencystretch\z@
  \tolerance 200%
  \hfuzz .1\p@
  \vfuzz\hfuzz}

\interfootnotelinepenalty=10000\relax

\usepackage[maxfloats=256]{morefloats}

\maxdeadcycles=500

\raggedbottomsectiontrue
\checkandfixthelayout

%%%
%\setlength{\parskip}{2cm plus0.5cm minus0.5cm}

%%%%%%%  biblatex
\newcommand{\noun}[1]{\textsc{#1}}    %  philosophy-verbose
\usepackage[backend=biber, sorting=nyt, style=verbose]{biblatex} %%%%ORIGINAL TiE
\renewcommand*{\mkbibnamefamily}[1]{\textsc{#1}}
% scauthors=all, publocformat=locpubyear,
% inbeforejournal=true, volnumformat=plain,relatedformat=parens
%\renewcommand*{\volnumpunct}{.}
%\addbibresource{GYS-Iet.bib}

\DeclareFieldFormat{url}{%
  \mkbibacro{URL}\addcolon\space
  \href{#1}{\nolinkurl{\thefield{urlraw}}}}

\DeclareFieldFormat{citeurl}{%
  \href{#1}{\nolinkurl{\thefield{urlraw}}}} 


%Index
%\usepackage[backend=biber, bibstyle=verbose, citestyle=authoryear]{biblatex}
%
%\DefineBibliographyStrings{german}{
%  references = {Literaturverzeichnis},
%  bibliography = {Bibliographie},
%  shorthands = {Abkürzungen der Zeugen des kritischen Apparatus},
%}

%\makeatletter
%\renewcommand*{\mkbibnamefamily}[1]{%
%  \ifdefstring{\blx@delimcontext}{parencite}
%    {\textsc{#1}}
%    {#1}}
%\makeatother

\DeclareFieldFormat{postnote}{#1}
\renewcommand{\postnotedelim}{, }
\addbibresource{bindu.bib}

%%% ekdosis
\usepackage[teiexport=tidy,parnotes=true]{ekdosis}% =tidy cleans up HTML and XML documents by fixing markup errors and upgrading legacy code to modern standards. parnotes=footnotes below or above critical apparatus

\SetLineation{lineation=page} %lineation=pagesets thenumbering to start afresh at the top of each page. modulo makes every fifth line numbered. 

\renewcommand{\linenumberfont}{\selectlanguage{english}\footnotesize} %sets language of lines to English

\SetTEIxmlExport{autopar=false} %autopar=falseinstructs ekdosis to ignore blank lines in the.tex sourcefile as markers for paragraph boundaries. As a result, each paragraph of the edition must be found within an environment associated with the xml <p> element

\SetHooks{
  lemmastyle=\bfseries,
  %refnumstyle=\selectlanguage{english}\bfseries,
  refnumstyle=\selectlanguage{english}\color{blue}\bfseries,
  appheight=0.8\textheight,
}

% Declare \ifinapparatus and set \inapparatustrue at the beginning of
% the apparatus criticus block. Also set the language.  
\newif\ifinapparatus
  \DeclareApparatus{default}[
  %bhook=\inapparatustrue, 
  lang=english,
  %maxentries=33,
  %bhook=\selectlanguage{english},
  sep = {] },
  delim=\hskip 0.75em,
  rule=\rule{0.7in}{0.4pt},
]

\newif\ifinapparatus
\DeclareApparatus{source}[
%bhook=\inapparatustrue,
lang=english,
notelang=english,
% bhook=\selectlanguage{english},
bhook=\selectlanguage{english}\textbf{Sources:},%
%maxentries=4, 
%ehook=.]
%sep={: },
]

\newif\ifinapparatus
\DeclareApparatus{testium}[
%bhook=\inapparatustrue,
lang=english,
notelang=english,
% bhook=\selectlanguage{english},
bhook=\selectlanguage{english}\textbf{Testimonia:},
%maxentries=4, 
%ehook=.]
%sep={: },
]

\newif\ifinapparatus
\DeclareApparatus{philcomm}[
%bhook=\inapparatustrue,
lang=english,
notelang=english,
bhook=\selectlanguage{english}\textbf{Philological Commentary:},
%bhook=\selectlanguage{english},
sep={: },
]

\ekdsetup{
showpagebreaks,
spbmk = \textcolor{blue}{spb},
hpbmk = \textcolor{red}{hpb}
}

% Macros and Definitions for the Print of Sigla
\def\acpc#1#2#3{{#1}\rlap{\textrm{\textsuperscript{#3}}}\textsubscript{\textrm{#2}}\space}
\def\sigl#1#2{{{#1}}\textsubscript{\textrm{#2}}}
\def\None{{\sigl{N}{1}}} \def\Noneac{\acpc{N}{1}{ac}\,} \def\Nonepc{\acpc{N}{1}{pc}\,}
\def\Ntwo{{\sigl{N}{2}}} \def\Noneac{\acpc{N}{2}{ac}\,} \def\Nonepc{\acpc{N}{2}{pc}\,}
\def\Done{{\sigl{D}{1}}} \def\Doneac{\acpc{D}{1}{ac}\,} \def\Donepc{\acpc{D}{1}{pc}\,}
\def\Dtwo{{\sigl{D}{2}}} \def\Dtwoac{\acpc{D}{2}{ac}\,} \def\Dtwopc{\acpc{D}{2}{pc}\,}
\def\Uone{{\sigl{U}{1}}} \def\Uoneac{\acpc{U}{1}{ac}\,} \def\Uonepc{\acpc{U}{1}{pc}\,}                 
\def\Utwo{{\sigl{U}{2}}} \def\Utwoac{\acpc{U}{2}{ac}\,} \def\Utwopc{\acpc{U}{2}{pc}\,}

%%%%%%%%%%%%%% Tattvabinduyoga - List of Witnesses   %%%%%%%%%%%%%%%%%%%
\DeclareWitness{ceteri}{\selectlanguage{english}cett.}{ceteri}[]   
\DeclareWitness{E}{\selectlanguage{english}E}{Printed Edition}[]    
\DeclareWitness{P}{\selectlanguage{english}P}{Pune BORI 664}[]  
\DeclareWitness{B}{\selectlanguage{english}B}{Bodleian 485}[]       
\DeclareWitness{N1}{\selectlanguage{english}N\textsubscript{1}}{NGMPP 38/31}[]
\DeclareWitness{N2}{\selectlanguage{english}N\textsubscript{2}}{NGMPP B 38/35}[]
\DeclareWitness{L}{\selectlanguage{english}L}{LALCHAND 5876}[]  
\DeclareWitness{D}{\selectlanguage{english}D}{IGNCA 30019}[] 
%\DeclareWitness{D2}{\selectlanguage{english}D\textsubscript{2}}{IGNCA 30020}[]  
\DeclareWitness{U1}{\selectlanguage{english}U\textsubscript{1}}{SORI 1574}[] 
\DeclareWitness{U2}{\selectlanguage{english}U\textsubscript{2}}{SORI 6082}[]
%%%%%%%%%%%%%% Tattvabinduyoga - Groups of Witnesses   %%%%%%%%%%%%%%%%%%%
\DeclareWitness{X}{\selectlanguage{english}\alpha}{Alpha Group: D,N1,N2,U1}[]
\DeclareWitness{Y}{\selectlanguage{english}\beta}{Beta Group: B,E,L,P,U2}[]
%%%%%%%%%%%%% Testimonia
\DeclareWitness{Ysv}{\selectlanguage{english}Ysv}{Yogasvarodaya}[] %%%add infos!  

%%%%%%%%%%%%%%%%%%%%%%%%%%%%%%%%%%%%%%%%%%%
% Macro for Editing Abbrevs.
\def\om{\textrm{\footnotesize \textit{om.}\ }} %prints om. for omitted in apparatus
\def\korr{\textrm{\footnotesize \textit{em.}\ }} %prints em. for emended in apparatus
\def\conj{\textrm{\footnotesize \textit{conj.}\ }} %prints conj. for conjectured in apparatus

% \supplied{text} EDITORIAL ADDITION -> Within \lem oder \rdg
% \surplus{text} EDITORIAL DELETION -> Within \lem oder \rdg
% \sic{text} CRUX
% \gap{text} LACUNAE -> [reason=??, unit=??, quantity=??, extent=??]


%%%%%%%%%%%%%%%%%%%%%%%%%%%%%%%%%%%%%%%%%%% All macros of this list can be used in 
% Macro for Editing Abbrevs.
\def\eyeskip{\textrm{{ab.\,oc. }}}
\def\aberratio{\textrm{{ab.\,oc. }}}
\def\ad{\textrm{{ad}}}
\def\add{\textrm{{add.\ }}}
\def\ann{\textrm{{ann.\ }}}
\def\ante{\textrm{{ante }}} 
\def\post{\textrm{{post }}}
%\def\ceteri{cett.\,}                   
\def\codd{\textrm{{codd.\ }}}

\def\coni{\textrm{{coni.\ }}}
\def\contin{\textrm{{contin.\ }}}
\def\corr{\textrm{{corr.\ }}}
\def\del{\textrm{{del.\ }}}
\def\dub{\textrm{{ dub.\ }}}

\def\expl{\textrm{{explic.\ }}} 
\def\explica t{\textrm{{explic.\ }}}
\def\fol{\textrm{{fol.\ }}}
\def\foll{\textrm{{foll.\ }}}
\def\gloss{\textrm{{glossa ad }}}
\def\ins{\textrm{{ins.\ }}}      
\def\inseruit{\textrm{{ins.\ }}} 
\def\im{{\kern-.7pt\lower-1ex\hbox{\textrm{\tiny{\emph{i.m.}}}\kern0pt}}} %\textrm{\scriptsize{i.m.\ }}}      
\def\inmargine{{\kern-.7pt\lower-.7ex\hbox{\textrm{\tiny{\emph{i.m.}}}\kern0pt}}}%\textrm{\scriptsize{i.m.\ }}}      
\def\intextu{{\kern-.7pt\lower-.95ex\hbox{\textrm{\tiny{\emph{i.t.}}}\kern0pt}}}%\textrm{\scriptsize{i.t.\ }}}           
\def\indist{\textrm{{indis.\ }}}  
\def\indis{\textrm{{indis.\ }}}
\def\iteravit{\textrm{{iter.\ }}} 
\def\iter{\textrm{{iter.\ }}}
\def\lectio{\textrm{{lect.\ }}}   
\def\lec{\textrm{{lect.\ }}}
\def\leginequit{\textrm{{l.n. }}} 
\def\legn{\textrm{{l.n. }}}
\def\illeg{\textrm{{l.n. }}}

\def\primman{\textrm{{pr.m.}}}
\def\prob{\textrm{{prob.}}}
\def\rep{\textrm{{repetitio }}}
\def\secundamanu{\textrm{\scriptsize{s.m.}}}            \def\secm{{\kern-.6pt\lower-.91ex\hbox{\textrm{\tiny{\emph{s.m.}}}\kern0pt}}}%   \textrm{\scriptsize{s.m.}}}
\def\sequentia{\textrm{{seq.\,inv.\ }}}  
\def\seqinv{\textrm{{seq.\,inv.\ }}}
\def\order{\textrm{{seq.\,inv.\ }}}
\def\supralineam{{\kern-.7pt\lower-.91ex\hbox{\textrm{\tiny{\emph{s.l.}}}\kern0pt}}} %\textrm{\scriptsize{s.l.}}}
\def\interlineam{{\kern-.7pt\lower-.91ex\hbox{\textrm{\tiny{\emph{s.l.}}}\kern0pt}}}   %\textrm{\scriptsize{s.l.}}}
\def\vl{\textrm{v.l.}}   \def\varlec{\textrm{v.l.}} \def\varialectio{\textrm{v.l.}}
\def\vide{\textrm{{cf.\ }}}
\def\cf{\textrm{{cf.\ }}} 
\def\videtur{\textrm{{vid.\,ut}}}
\def\crux{\textup{[\ldots]} }
\def\cruxx{\textup{[\ldots]}}
\def\unm{\textit{unm.}}
%%%%%%%%%%%%%%%%%%%%%%%%%%%%%%%%%%%%

% List of Scholars
\DeclareScholar{ego}{ego}[
forename=Nils Jacob,
surname=Liersch]

% Persons:14\DeclareScholar{ego}{ego}[15forename=Robert,16surname=Alessi]17% Useful shorthands:18\DeclareShorthand{codd}{codd.}{V,I,R,H}19\DeclareShorthand{edd}{edd.}{Lit,Erm,Sm}20\DeclareShorthand{egoscr}{\emph{scripsi}}{ego}

%Useful shorthands:
%\DeclareShorthand{codd}{codd.}{V,I,R,H}
%\DeclareShorthand{edd}{edd.}{Lit,Erm,Sm}
\DeclareShorthand{egoscr}{em.}{ego}
\DeclareShorthand{egoscrconj}{conj.}{ego}
\DeclareShorthand{egomute}{\unskip}{ego}

\usepackage{xparse}

\NewDocumentEnvironment{tlg}{O{}O{}}{\setlength{\leftskip}{0pt}\vspace{-1ex}\begin{quotation}}{\hfill #1\ \vspace{-1ex}\end{quotation}\vspace{-1ex}} %verse environment
%\NewDocumentEnvironment{tlg}{O{}O{}}{\begin{verse}}{॥#1\hskip-4pt ॥\\ \end{verse}}
\NewDocumentCommand{\tl}{m}{{\selectlanguage{iast} #1}}

\NewDocumentCommand{\extra}{m}{{\textcolor{MidnightBlue}{#1}}} %command for additions to U2
\NewDocumentCommand{\crazy}{m}{{\textcolor{red}{#1}}} %totally corrupted passage
\NewDocumentCommand{\coro}{m}{{\textcolor{violet}{#1}}} %colour for sentence counter! 

\NewDocumentEnvironment{prose}{O{}}{\begin{otherlanguage}{iast}}{\end{otherlanguage}}
% \NewDocumentEnvironment{padd}{O{}}{\begin{otherlanguage}{iast}}{\end{otherlanguage}}
\NewDocumentEnvironment{tlate}{O{}}
%\NewDocumentEnvironment{tadd}{O{}}

%Define two commands: \skp ("sanskrit plus"), to be ignored by TeX in
%the edition text, but processed in the TEI output. Conversely, \skm
%("sanskrit minus") is to be processed in the edition text, but
%ignored if found in the apparatus criticus and in the TEI output:

\NewDocumentCommand{\skp}{m}{}
\TeXtoTEIPat{\skp {#1}}{#1}

%\NewDocumentCommand{\skpp}{m}{}
%\TeXtoTEIPat{\skpp {#1}}{#1}

\NewDocumentCommand{\skm}{m}{\unless\ifinapparatus#1-\fi}
\TeXtoTEIPat{\skm {#1}}{}

\NewDocumentCommand{\dd}{}{/\hskip-4pt/}
\TeXtoTEIPat{\dd {}}{//}


%%% modify environments and commands
%%% TEI mapping
\TeXtoTEIPat{\begin {tlg}}{<lg>} %lg=(Group of verse (s)) contains one or more verses or lines of verse that together form a formal unit (e.g. stanza, chorus).
\TeXtoTEIPat{\end {tlg}}{</lg>}

\TeXtoTEIPat{\begin {prose}}{<p>}
\TeXtoTEIPat{\end {prose}}{</p>}

\TeXtoTEIPat{\begin {tlate}}{<p>}
\TeXtoTEIPat{\end {tlate}}{</p>}

\TeXtoTEIPat{\\}{}
\TeXtoTEIPat{\linebreak}{<br/>}
\TeXtoTEIPat{\noindent}{}
%\TeXtoTEI{tl}{l}
\TeXtoTEI{emph}{hi}
\TeXtoTEI{bigskip}{}
\TeXtoTEI{None}{N1}
\TeXtoTEI{Ntwo}{N2}
\TeXtoTEI{Done}{D1}
\TeXtoTEI{Dtwo}{D2}
\TeXtoTEI{Uone}{U1}
\TeXtoTEI{Utwo}{U2}
%\TeXtoTEIPat{/}{ |}
%\TeXtoTEI{//}{ ||}
\TeXtoTEIPat{\korr}{em. }
\TeXtoTEIPat{\conj}{conj.}
\TeXtoTEIPat{\om}{om.}
\TeXtoTEIPat{english}{}
\TeXtoTEIPat{\hskip}{}
\TeXtoTEIPat{\hskip-4pt}{}
\TeXtoTEIPat{\hskip-2pt}{}
\TeXtoTEIPat{-}{ }
\TeXtoTEIPat{4pt}{}
\TeXtoTEIPat{2pt}{}
\TeXtoTEIPat{\textcolor {#1}{#2}}{<hi rend="#1">#2</hi>} 

% Nullify \selectlanguage in TEI as it has been used in
% \DeclareWitness but should be ignored in TEI.
\TeXtoTEI{selectlanguage}{}
\author{Nils Jacob Liersch}
\title{Minimalbeispiel Alignment}
\date{\today}

\parindent=15pt
\begin{document}

% Zitiermöglichkeiten:
%\footcite[See][p.\,1]{goldstein01:_tibet_englis_diction_moder_tibet}
%\footnote{\cite{goldstein01:_tibet_englis_diction_moder_tibet}.}

\frontmatter
\thispagestyle{empty}
\begin{center}
  {\Large \emph{The Yogatattvabindu}}\\[3mm]
\end{center}



\newpage

\

\thispagestyle{empty}



\normalsize


\newpage


\begin{center}
\thispagestyle{empty}

\

\vskip 2mm

\begin{otherlanguage}{iast}
\LARGE \sanskritfont{Yogatattvabindu}
\end{otherlanguage}

\vskip .4cm

\Huge Yogatattvabindu \\[7mm]
\Large Critical and Synoptic \\
Edition with annotated Translation


\large

\vspace{3cm}

Von

Nils Jacob Liersch
\small
\vfill

\vfill

Indica et Tibetica Verlag \\ % $\cdot$ 
Marburg 2024

\vskip 6mm

\end{center}

\newpage
\newpage \ \thispagestyle{empty}
\small  \

\noindent

\
\vfill


\small
\noindent \textbf{Bibliographische Information Der Deutschen Bibliothek}

\noindent
Die Deutsche Bibliothek verzeichnet diese Publikation in der Deutschen Nationalbibliographie;
detaillierte bibliographische Informationen sind im Internet über http://dnb.ddb.de abrufbar.

\noindent
\textbf{Bibliographic information published by Die Deutschen Bibliothek}

\noindent
Die Deutsche Bibliothek lists this publication in the Deutsche Nationalbibliographie; detailed
bibliographic data is available in the Internet at http://dnb.ddb.de.  


\vskip 1cm

\noindent
\copyright\ Indica et Tibetica Verlag, Marburg 2024

\medskip

\noindent
Alle Rechte vorbehalten / All rights reserved

\medskip

\noindent
Ohne ausdrückliche Genehmigung des Verlages ist es nicht gestattet, das Werk oder einzelne Teile
daraus nachzudrucken, zu vervielfältigen oder auf Datenträger zu speichern.

\smallskip

\noindent
Apart from any fair dealing for the purpose of private study, research, criticism or review, no
part of this book may be reproduced or translated in any form, by print, photo form, microfilm, or
any other means without written permission. Enquiries should be made to the publishers.

\bigskip

\noindent
Satz: \ \ Nils Jacob Liersch \\
Herstellung: \ \ BoD – Books on Demand GmbH, Norderstedt  \\

\bigskip

\noindent
%\ISBN     

\normalsize

\newpage

%\maketitle
\clearpage
\tableofcontents
\addtocounter{page}{-1}
\thispagestyle{empty}
\clearpage

\chapter{Introduction}
\mainmatter

\chapter{Critical Edition \& Annotated Translation}
\cleardoublepage
\begin{alignment}[
  texts=edition[class="edition"];
  translation[class="translation"],
  ]
  \begin{edition}
    \ekddiv{type=ed}
    \begin{prose}
      \noindent
\note[type=source, labelb=178a, lem={\textbf{Re}}]{PT\textsuperscript{qcr \cdot YSV} (Ed. pp. 838-839): netramadhye kūrmanāmā nimeṣonmeṣakṛdayam | udgāre nāga ākhyātaḥ ūrddhavāyuḥ pracālane | kṛkaraḥ kṣutkaro jñeyo devadatto vijṛmbhaṇe | dhanañcayaḥ saccidākāro mṛtadehaṃ na muñcati | yady api sargakāṇḍe sarvametaduktaṃ tathāpi kāryakāraṇabhāvajñāpanāya punarnirdiṣṭamiti na punaruktam |}
\note[type=source, labelb=179a, lem={\textbf{Ri}}]{SSP 1.67 (Ed. pp. 23-24): kūrmavāyuḥ cakṣuṣor unmeṣakārakaś ca| kṛkalaḥ udgārakaḥ kṣutkārakaś ca | devadatto mukhavijṛmbhakaḥ | dhanañjayo nādaghoṣakah ||1.67|| iti daśavāyvavalokanena piṇḍotpattiḥ naranārīrūpam |}
%-----------------------------
%kūrmavāyur netramadhye tiṣṭhati/ nimeṣonmeṣaṃ karoti/ \E
%kūrmavāyur netramadhye           nimeṣonmeṣaṃ karoti \P
%kūrmavāyoḥ netramadhye           nimeṣonmeṣaṃ karotī/ \B
%kūrmavāyoḥ netramadhye           nimiṣonmeṣaṃ karotī... \L
%kūrmo vāyunetramadhye tiṣṭhati/  unmeṣaṃ nimeṣaṃ karoti/ \N1
%kūrmo vāyunetramadhye tiṣṭhati/  unmeṣaṃ nimeṣaṃ ca karoti// \D
%kūrmo vāyunetramadhye tiṣṭhati/  unmeṣaṃ nimeṣaṃ karoti/ \N2
%\om                                                     \U1
%kūrmavāyur netramadhye           nimiṣonmeṣaṃ karoti//            \U2
%-----------------------------
%The Kūrma vitalwind exists within the eyes. It causes [the] opening and closing [of the eyes]. 
%-----------------------------
\app{\lem[wit={E,P,U2}, alt={kūrmavāyur}]{kūrmavāyu\skp{r-ne}}
  \rdg[wit={B,L}]{kūrmavāyoḥ}
  \rdg[wit={D,N1,N2}]{kūrmo vāyu}
}\skm{r-ne}tramadhye
\app{\lem[wit={D,E,N1,N2}]{tiṣṭhati}
  \rdg[wit={ceteri}]{\om}}/  
\app{\lem[wit={E,P,B,U2}]{nimeṣonmeṣaṃ}
  \rdg[wit={N1,N2}]{unmeṣaṃ nimeṣaṃ}
  \rdg[wit={D}]{unmeṣaṃ nimeṣaṃ ca}}
\app{\lem[wit={ceteri}]{karoti}
  \rdg[wit={B,L}]{karotī}}/
\note[type=philcomm, labelb=179b, lem={\uproman{27}.\textsuperscript{\lowroman{17}-\lowroman{18}}}]{Sentences \om in \getsiglum{U1}.}
%-----------------------------
%kṛkalakartāvāyur  udgāraṃ karoti      \E
%kṛkalavāyur       udhāraṃ karoti      \P
%kṛkalavāyur       udhāraṃ karotī      \B
%kṛkalavāyur       uhāraṃ karotī        \L
%kṛkalavāyor       ūdgāro bhavati//    \N1
%kṛkalavāyor-------ūdgāto bhavati/      \D
%kṛkaravāyor-------ūdgāro bhavati/      \N2
%                                       \U1
%puṣkaravāyur      udgāraṃ karoti//    \U2
%-----------------------------
%From the Kṛkala vitalwind gagging arises. 
%-----------------------------
\app{\lem[wit={D,N1,N2},alt={kṛkalavāyor}]{kṛkalavāyo\skp{r-u}}
  \rdg[wit={B,L,P}]{kṛkalavāyur}
  \rdg[wit={E}]{kṛkalakartāvāyur}
  \rdg[wit={U2}]{puṣkaravāyur}
}\app{\lem[type=emendation, resp=egoscr, alt={udgāro}]{\skm{r-u}dgāro}
  \rdg[wit={E,U2}]{udgāraṃ}
  \rdg[wit={B,P}]{udhāraṃ}
  \rdg[wit={L}]{uhāraṃ}
  \rdg[wit={N1,N2}]{ūdgāro}
  \rdg[wit={D}]{ūdgāto}}
\app{\lem[wit={D,N1,N2}]{bhavati}
  \rdg[wit={E,P,U2}]{karoti}
  \rdg[wit={B,L}]{karotī}}/
%-----------------------------
% devadattavāyoḥ  jṛmbhaṇaṃ bhavati/ dhanaṃjayavāyoḥ śabda utpadyate// \E
% devadattavāyor  jumbhā bhavati     dhanaṃjayavāyo  śabdāḥ utpadyete  \P
% devadattavāyor  jumbhā bhavaṃtī    dhanaṃjayavāyoḥ śabda utpadyate// \B
% devadattavāyor  jṛṃbhā bhavatī     dhanaṃjayavāyoḥ śabdaḥ utpadyate// \L
% devadattavāyor  jṛṃbha utpadyate// dhanaṃjayavāyo  śabda utpadyate// \N1
% devadattavāyor  jṛṃbha utpadyate// dhanaṃjayavāyo  śabda utpadyate// \D
% devadattavāyo   jṛṃbhotpadyate/    dhanaṃjayavāyo  śabdotpadyate// \N2
% devadattavāyor  jaṃbhā utpadyate   dhanaṃjayavāyoḥ sabta utpadyate \U1
% devadattavāyo   jṛṃbhā bhavati//   dhanaṃjayavāyoḥ śabda utpadyate// \U2
%-----------------------------
%From the Devadatta vitalwind jawning arises. From the Dhanaṃjaya vitalwind speech arises. 
%-----------------------------
\app{\lem[wit={ceteri}, alt={devadattavāyor}]{devadattavāyo\skp{r-jṛ}}
  \rdg[wit={E}]{devadattavāyoḥ}
  \rdg[wit={N2,U2}]{devadattavāyo}
}\app{\lem[wit={D,N1,U2},alt={jṛmbha}]{\skm{r-jṛ}mbha}
  \rdg[wit={E}]{jṛmbhaṇaṃ}
  \rdg[wit={B,P}]{jumbhā}
  \rdg[wit={L}]{jṛṃbhā}
  \rdg[wit={N2}]{jṛṃbho°}
  \rdg[wit={U1}]{jaṃbhā}}
\app{\lem[wit={X}]{utpadyate}
  \rdg[wit={E,P,U2}]{bhavati}
  \rdg[wit={B}]{bhavaṃtī}
  \rdg[wit={L}]{bhavatī}}/
\app{\lem[wit={Y}]{dhanaṃjayavāyoḥ}
  \rdg[wit={X}]{dhanaṃjayavāyo}}
\app{\lem[wit={ceteri}]{śabda}
  \rdg[wit={P}]{śabdāḥ}
  \rdg[wit={L}]{śabdaḥ}
  \rdg[wit={N2}]{śabdo°}
  \rdg[wit={U1}]{sabta}}
utpadyate\dd{}\textsuperscript{\begin{otherlanguage}{english}\coro{[\lowroman{20}]}\end{otherlanguage}}\vfill
\end{prose}
\nolinenumbers
\centerline{\textrm{\small{[\uproman{27}.\textsuperscript{\coro{\lowroman{1}-\lowroman{5}}}Madhyalakṣya]}}}
\label{madhyalaksya}
    \bigskip
    \linenumbers
    \begin{prose}
      \noindent
%----------------------------
%\om                               \E
%idānī  madhyalakṣaṃ   kathyate      \P
%idānīṃ madhyalakṣaṇaṃ kathyate//  \B DSCN7164 Z.1
%idānīṃ madhye lakṣaṃ  kathyate//   \L
%idānīṃ madhyalakṣyaṃ  kathyate//   \N1
%idānīṃ madhyalakṣyaṃ  kathyate//   \D
%idānīṃ madhyalakṣaṇaṃ kathyate//  \N2
%idānīṃ madhyalakṣyaṃ  kathyate     \U1
%idānīṃ madhye lakṣyaṃ kathyate//  \U2
%-----------------------------
%Now the central fixation is taught. 
%-----------------------------
\note[type=source, labelb=180, lem={\textbf{Re}}]{PT\textsuperscript{qcr \cdot YSV} (Ed. p. 839): idānīṃ madhyalakṣan tu kathyate siddhikārakam | śvetaṃ raktaṃ tathā pītaṃ dhūmrākāran tu nīlabham |}
\app{\lem[wit={ceteri}]{idānīṃ}
  \rdg[wit={P}]{idānī}}
\app{\lem[wit={D,N1,U1}]{madhyalakṣyaṃ}
  \rdg[wit={B,N2}]{madhyalakṣaṇaṃ}
  \rdg[wit={P}]{madhyalakṣaṃ}
  \rdg[wit={L}]{madhye lakṣaṃ}
  \rdg[wit={U2}]{madhye lakṣyaṃ}}
kathyate/\note[type=philcomm, labelb=180a, lem={\uproman{27}\textsuperscript{\lowroman{1}}}]{Introductory sentence is missing in \getsiglum{E}.}
%-----------------------------SSP. S41!!! almost identical! 
%              aṃtha ca pītavarṇaṃ   raktavarṇaṃ vā dhūmrākāraṃ yan  nīlavarṇaṃ vā   agniśikhāsadṛśaṃ vidyutsamānaṃ   sūryamaṇḍalasadṛśaṃ     arddhacandrasadṛśaṃ jvalad  ākāśasamākāraṃ  \E
%śvetavarṇaṃ   atha     pītavarṇaṃ   raktaṃ vā      dhūmrākāraṃ yan  nīlavarṇaṃ vā   'gniśikhāsadṛśaṃ vidyutsamānaṃ   sūryamaṇdalasadṛśaṃ     arddhacaṃdrasadṛśaṃ jvalad  ākāśasamākāraṃ  \P
%śvetavaraṃ    atha     pītavarṇaṃ// rakta  vā      dhūmrākāraṃ yan  nīlavarṇaṃ vā// agniśikhāsadṛśaṃ vidyutsamānaṃ   sūryamaṇdalasadṛśaṃ/    ūrdhvacaṃdrasadṛśaṃ jvalad  ākāśasamākāraṃ// \B
%śvetavarṇaṃ   atha     pītavarṇaṃ   raktaṃ vā      dhūmrākāraṃ yan  nīlavarṇaṃ vā// agniśikhāsadṛśaṃ vidyutsamāne    sūryamaṇdalasadṛśaṃ//   ardhacaṃdrasadṛśaṃ  jvalad  ākāśasamākāra   \L
%śvetavarṇā/   atha vā  pītavarṇaṃ   raktaṃ vā      dhūmāra     va   nīlavarṇaṃ vā   agniśikhāsadṛśaṃ vidyutsamānaṃ   sūryamaṇdalaṃ sadṛśaṃ/  ūrdhvacaṃdrasadṛśaṃ jvalad  ākāśasamānakāraṃ//  \N1
%śvetavarṇaṃ// atha vā  pītavarṇaṃ   raktaṃ vā      dhūmākāro   vā   nīlavarṇaṃ vā   agniśikhāsadṛśaṃ vidyutsamānaṃ// sūryamaṇdalaṃ sadṛśaṃ// ūrdhvacaṃdrasadṛśaṃ jvalad  ākāśasamānakāraṃ//  \D
%śvetavarṇā    atha vā  pītavarṇa    raktavarṇa     dhūmravarṇa      nīlavarṇaṃ vā   agniśikhāsadṛśaṃ vidyutsamānaṃ   sūryamaṇdalasadṛśaṃ     ūrdhvacaṃdrasadṛśaṃ jvalad  ākāśasamānakāraṃ//  \N2
%svetavarṇaṃ   atha vā  pītavarṇaṃ   raktaṃ vā      dhūmrākāra  van  nīlavarṇaṃ vā   agniśikhāsadṛśaṃ vidyutsamānaṃ   sūryamaṇdalasadṛśaṃ     ārdhacaṃdrasadṛśaṃ  jalad---ā----samānākāraṃ \U1
%svatavarṇaṃ   atha vā  pītavarṇaṃ// raktaṃ vā      dhūmrākāraṃ yan  nīlavarṇaṃ vā   agniśikhāsadṛśaṃ vidyutsamānaṃ   sūryamaṇdalasadṛśaṃ     arddhacaṃdrasadṛśaṃ jvalad--ākāraṃ samākāraṃ \U2
%-----------------------------
%White-colored, or also yellow-colored or red-coloured or smoke-coloured or blue-coloured, like the flame of fire, equal to a lightning, like the orb of the sun, like a half-moon, appearing like flaming space, ...  
%-----------------------------
\note[type=source, labelb=181, lem={\textbf{Re}}]{PT\textsuperscript{qcr \cdot YSV} (Ed. p. 839): agnijvālāsamānābhā vidyutpuñjasamaprabhā | ādityamaṇḍalākāramathavā candramaṇḍalam |}
\note[type=source, labelb=181a, lem={\textbf{Ri}}]{SSP 2.29 (Ed. p. 41): śvetavarṇaṃ vā raktavarṇaṃ vā kṛṣṇavarṇaṃ vā agniśikhākāraṃ vā jyotirūpaṃ vā vidyudākāraṃ sūryamaṇḍalākāraṃ vā arddhacandrākāraṃ vā yatheṣṭasvapiṇḍamātraṃ sthānavarjitaṃ manasā lakṣayet ity anekaviddhaṃ madhyamaṃ lakṣyaṃ |}
\app{\lem[wit={ceteri}, alt={°śveta}]{śveta}
  \rdg[wit={U1}]{sveta°}
  \rdg[wit={U2}]{svata°}
  \rdg[wit={E}]{\om}
}\app{\lem[wit={P,L,U1,U2}, alt={°varṇaṃ}]{varṇaṃ}
  \rdg[wit={D}]{°varṇaṃ ||}
  \rdg[wit={P}]{°varaṃ}
  \rdg[wit={N1}]{°varṇā |}
  \rdg[wit={E}]{\om}}
\app{\lem[wit={ceteri}]{atha}
  \rdg[wit={E}]{aṃtha}}
\app{\lem[wit={ceteri}]{vā}
  \rdg[wit={E}]{ca}
  \rdg[wit={B,L,P}]{\om}}
pīta\app{\lem[wit={ceteri}, alt={°varṇaṃ}]{varṇaṃ}
  \rdg[wit={B,U2}]{°varṇaṃ ||}
  \rdg[wit={N2}]{°varṇa}}
\app{\lem[wit={E}, alt={raktavarṇaṃ}]{raktavarṇaṃ}
  \rdg[wit={N2}]{raktavarṇa}
  \rdg[wit={D,L,N1,U1,U2}]{raktaṃ}
  \rdg[wit={B}]{\om}}
\app{\lem[wit={ceteri}]{vā}
  \rdg[wit={N2}]{\om}}
\app{\lem[type=emendation, resp=egoscr]{dhūmravarṇaṃ}
  \rdg[wit={D}]{dhūmākāro}
  \rdg[wit={N1}]{dhūmāra}
  \rdg[wit={N2}]{dhūmravarṇa}
  \rdg[wit={U1}]{dhūmrākāra}
  \rdg[wit={Y}]{dhūmrākāraṃ}}
%\note[type=philcomm, labelb=182, lem={dhūmra°}]{Due to the repetitive use of the term \textit{varṇaṃ}, both preceding and succeeding the mention of \textit{dhūmra}, as well as its previous occurrence within the same compound, it is highly probable that the original reading was \textit{dhūmravarṇaṃ}.}
\app{\lem[wit={D}]{vā}
  \rdg[wit={N1}]{va}
  \rdg[wit={U1}]{van}
  \rdg[wit={Y}]{yan}
  \rdg[wit={N2}]{\om}}
nīlavarṇaṃ
\app{\lem[wit={ceteri}]{vā}
  \rdg[wit={B,L}]{vā ||}}
\app{\lem[wit={P}, alt={'gni°}]{'gni}
  \rdg[wit={ceteri}]{agni°}
}śikhāsadṛśaṃ vidyut\app{\lem[wit={ceteri},alt={°samānaṃ}]{samānaṃ}
  \rdg[wit={D}]{°samānaṃ ||}
  \rdg[wit={L}]{°samāne}}
sūryamaṇdala\app{\lem[wit={ceteri}, alt={°sadṛśaṃ}]{sadṛśaṃ}
  \rdg[wit={D,N1}]{°ṃ sadṛśaṃ}}
\app{\lem[wit={ceteri},alt={ardha°}]{ardha}
  \rdg[wit={B,D,N1,N2}]{ūrdhva°}
  \rdg[wit={U1}]{ārdha°}
}candrasadṛśaṃ
\app{\lem[wit={ceteri}, alt={jvalad°}]{jvala\skp{d-ā}}
  \rdg[wit={U1}]{jalad}
}\app{\lem[wit={ceteri}, alt={°ākāśa°}]{\skm{d-ā}kāśa}
  \rdg[wit={U1}]{°ā°}
  \rdg[wit={U2}]{°ākāraṃ}
}\app{\lem[wit={ceteri}, alt={°samākāraṃ}]{samākāraṃ}
  \rdg[wit={D,N1,N2,U1}]{°samānakāraṃ}
  \rdg[wit={U2}]{samakāraṃ}
  \rdg[wit={L}]{°samākāra}}/
%-----------------------------
%svaśarīraparimitaṃ      tejomanomadhye tathyaṃ kartavyam// \E
%svaśarīraparimitaṃ      tejomanomadhye lakṣyaṃ karttavyaṃ\P
%svaśarīraparimitaṃ      tejomanomadhye lakṣaṃ kartavyaṃ//  \B
%svaśarīraparimitaṃ      tejomanomadhye lakṣaṃ kartavyaṃ//  \L
%svaśarīraparimitaṃ      tejomanomadhye lakṣyaṃ karttavyaṃ//  \N1
%svaśarīraparimitaṃ      tejomanomadhye lakṣyaṃ karttavyaṃ// \D
%svaśarīraparimitaṃ      tejomanomadhye lakṣaṇaṃ karttavyaṃ//  \N2
%svaśarīraparimanomittaṃ tejomadhye     lakṣyaṃ karttavyaṃ  \U1
%svaśarīraparimitaṃ      tejomanomadhye lakṣaṃ kartavyaṃ//  \U2
%-----------------------------
%measured according to ones own body, the fixation shall be directed onto the center of the glowing mind.  
%-----------------------------
\note[type=source, labelb=183, lem={\textbf{Re}}]{PT\textsuperscript{qcr \cdot YSV} (Ed. p. 839): jvaladākāśatulyaṃvā bhāvayed rūpamātmanaḥ | etaj jyotirmayaṃ dehaṃ manomadhye tu lakṣayet |}
svaśarīrapari\app{\lem[wit={ceteri},alt={°mitaṃ}]{mitaṃ}
  \rdg[wit={U1}]{°manomittaṃ}}
tejo\app{\lem[wit={ceteri},alt={°mano}]{mano}
  \rdg[wit={U1}]{\om}
}madhye
\app{\lem[wit={D,P,N1,U1}]{lakṣyaṃ}
  \rdg[wit={E}]{tathyaṃ}
  \rdg[wit={B,L,U2}]{lakṣaṃ}
  \rdg[wit={N2}]{lakṣaṇaṃ}}
kartavyaṃ/
%-----------------------------
%ekasmin lakṣye    kṛte sati manomadhye sthitasya malasya dāho bhavati/ \E [p.38]%
%etasmil lakṣye    kṛte sati manomadhye sthitasya         dāho bhavati \P
%etasmin lakṣe     kṛte satī manomadhye sthitasya malasya dāho bhavati \B
%etasmil lakṣe     kṛte satī manomadhye sthitasya malasya dāho bhavati... \L
%ekasmin lakṣye    kṛte sati manomadhye sthitasya malasya dāho bhavati/ \N1
%ekasmin lakṣye    kṛte sati manomadhye sthitasya malasya dāho bhavati// \D
%ekasmin lakṣaṇo   kṛte sati manomadhye sthitasya malasya dāho bhavati/ \N2
%etasmin na lakṣye kṛte satī manomadhye sthitasya malasya dāho bhavati \U1 %%%281.jpg
%etasmil lakṣe     kṛte satī manomadhye sthitasya malasya dāho bhavati// \U2
%-----------------------------
%While abiding in this fixation the burning of the impurity in the center of the mind arises. 
%-----------------------------
\note[type=source, labelb=184, lem={\textbf{Re}}]{PT\textsuperscript{qcr \cdot YSV} (Ed. p. 839): eteṣāñ ca kṛte lakṣe nānāduḥkhaṃ praṇaśyati | manas astu malo yāti mahānando bhavet tataḥ |}
\app{\lem[wit={P,L,U2},alt={etasmil}]{etasmi\skp{ll-a}}
  \rdg[wit={U1}]{etasmin}
  \rdg[wit={ceteri}]{ekasmin}
}\app{\lem[wit={ceteri},alt={lakṣye}]{\skm{ll-a}kṣye}
  \rdg[wit={B,L,U2}]{lakṣe}
  \rdg[wit={U1}]{na lakṣye}
  \rdg[wit={N2}]{lakṣaṇo}}
kṛte
\app{\lem[wit={ceteri}]{sati}
  \rdg[wit={B,L,U1,U2}]{satī}}
manomadhye sthitasya
\app{\lem[wit={ceteri}]{malasya}
  \rdg[wit={P}]{\om}}
dāho bhavati/ 
%-----------------------------
%manasaḥ   sattvaguṇaprakāśo   bhavati/     puruṣa ānandamayo bhūtvā tiṣṭhati//   \E 
%manasaḥ   sattvaguṇaḥ prakaṭo bhavati      puruṣa ānandamayo bhūtvā tiṣṭhati    \P   %%%7650.jpg
%manasaḥ// sattvaguṇo  prakaṭo  bhavati//   puruṣa ānandamayo bhūtvā tiṣṭhati// \B
%manasaḥ// sattvaguṇaḥ prakaṭo bhavati      puruṣa ānandamayo bhūtvā tiṣṭhati//  \L
%manasaḥ   sattvaguṇe  prakaṭo  bhavati/    puruṣa ānandamayo bhūtvā tiṣṭhati//    \N1
%manaḥ saḥ sattvaguṇo  prakaṭo  bhavati//   puruṣa ānandamayo bhūtvā tiṣṭhati// \D
%manasaḥ   sattvaguṇo  prakaṭo  bhavati/    puruṣa ānandamayo bhūtvā tiṣṭhati//    \N2
%manasaḥ   sattvaguṇo  prakaṭo  bhavati     puruṣa ānandamayo bhūtvā tiṣṭhati       \U1
%manasaḥ   satvaguṇaprakaśo    bhavati//    puruṣa ānandamayo bhūtvā tiṣṭhati//    \U2 %%414.jpg
%-----------------------------
%The Sattva-quality of the mind becomes revealed. After this has happend the person abides supreme bliss. 
%-----------------------------
mana\app{\lem[wit={ceteri},alt={°saḥ}]{saḥ}
  \rdg[wit={B,L}]{°saḥ ||}
  \rdg[wit={D}]{manaḥ saḥ}}
sattva\app{\lem[wit={B,D,N2,U1},alt={°guṇo}]{guṇo}
  \rdg[wit={N1}]{°guṇe}
  \rdg[wit={E,U2}]{°guṇa°}
  \rdg[wit={P,L}]{°guṇaḥ}}
\app{\lem[wit={ceteri}]{prakaṭo}
  \rdg[wit={E,U2}]{°prakāśo}}
bhavati/ puruṣa ānandamayo bhūtvā tiṣṭhati\dd{}\textsuperscript{\begin{otherlanguage}{english}\coro{[\lowroman{5}]}\end{otherlanguage}}\vfill
\end{prose}
  \end{edition}
  \begin{translation}
    \ekddiv{type=trans}
    \begin{tlate}
\noindent
The Kūrma vital wind exists within the eyes. It causes [the] opening and closing [of the eyes]. From the Kṛkala vital wind gagging arises. From the Devadatta vital wind jawning arises. From the Dhanaṃjaya vital wind speech arises.\textsuperscript{\coro{[\lowroman{20}]}}
\end{tlate}
  \ekddiv{type=trans}
  \bigskip
\centerline{\textrm{\small{[\uproman{26}.\textsuperscript{\coro{\lowroman{1}-\lowroman{5}}}Madhyalakṣya]}}}
\bigskip
\begin{tlate}
Now the central fixation is taught. White-coloured or also yellow-coloured or red-coloured or smoke-coloured or blue-coloured, like the flame of fire, equal to a lightning, like the orb of the sun, like a half-moon, appearing like flaming space, measured according to one's own body, the fixation shall be directed onto the centre of the glowing mind.\footnote{\textit{Śivayogapradīpikā} 4.47cd-48: \begin{quote} śṛṇuṣva madhyalakṣyaṃ ca kathitaṃ pūrvasūribhiḥ || 4.47 \\
 śvetādivarṇanavakhaṇḍasucandrasūryasaudāminīvahniśikhena bimbāt | \\
 jvalan nabho vā sthalahīnam ekaṃ vilakṣayet tat khalu madhyalakṣyam 4.48 ||
 \end{quote}
``(47cd) Hear now the central target taught by the ancient sages. \\
(48) Through the image consisting of rays of fire, lightning, sun, moon, and nine [different] colors such as white, etc., one should fixate on the luminous ether or on that which is locationless. Verily, this is the central goal.''
}
While abiding in the fixation, the burning of the impurity in the centre of the mind arises. The Sattva quality of the mind becomes revealed.\footnote{The generation of the sattvic quality through the practice of \textit{madhyalakṣ(y)a} also appears in \citetitle{sarvangayoga} 3.28: \begin{quote}madhya lakṣa mana madhya bicārai | vapu pramāna koi rūpa nihārai |\\
yāte sātvik upajai āī | madhya lakṣa jo sādhai bhāī || 28 ||\end{quote} ``The central Lakṣa directs the mind to reside at its center, revealing the true form of the body. It produces a sattvic quality in those who practice it.'' (28)} After this has happened, the person abides in supreme bliss.   
\end{tlate}
  \end{translation}
\end{alignment}
\chapter{Bibliography}
 \label{sec:bibli}
   \clearpage
\newpage 
\thispagestyle{empty}
\quad  \addtocounter{page}{-1}

\printbibliography[heading=subbibintoc, title=Consulted Manuskripts, keyword=codex]

\printbibliography[heading=subbibintoc, title=Printed Editions, keyword=printsource]

\printbibliography[heading=subbibintoc, title=Secondary Literature, keyword=seclit]

\printbibliography[heading=subbibintoc, title=Online Sources, keyword=onlinesource]

\end{document}

